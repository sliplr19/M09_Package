\documentclass[11pt]{article}

    \usepackage[breakable]{tcolorbox}
    \usepackage{parskip} % Stop auto-indenting (to mimic markdown behaviour)
    

    % Basic figure setup, for now with no caption control since it's done
    % automatically by Pandoc (which extracts ![](path) syntax from Markdown).
    \usepackage{graphicx}
    % Maintain compatibility with old templates. Remove in nbconvert 6.0
    \let\Oldincludegraphics\includegraphics
    % Ensure that by default, figures have no caption (until we provide a
    % proper Figure object with a Caption API and a way to capture that
    % in the conversion process - todo).
    \usepackage{caption}
    \DeclareCaptionFormat{nocaption}{}
    \captionsetup{format=nocaption,aboveskip=0pt,belowskip=0pt}

    \usepackage{float}
    \floatplacement{figure}{H} % forces figures to be placed at the correct location
    \usepackage{xcolor} % Allow colors to be defined
    \usepackage{enumerate} % Needed for markdown enumerations to work
    \usepackage{geometry} % Used to adjust the document margins
    \usepackage{amsmath} % Equations
    \usepackage{amssymb} % Equations
    \usepackage{textcomp} % defines textquotesingle
    % Hack from http://tex.stackexchange.com/a/47451/13684:
    \AtBeginDocument{%
        \def\PYZsq{\textquotesingle}% Upright quotes in Pygmentized code
    }
    \usepackage{upquote} % Upright quotes for verbatim code
    \usepackage{eurosym} % defines \euro

    \usepackage{iftex}
    \ifPDFTeX
        \usepackage[T1]{fontenc}
        \IfFileExists{alphabeta.sty}{
              \usepackage{alphabeta}
          }{
              \usepackage[mathletters]{ucs}
              \usepackage[utf8x]{inputenc}
          }
    \else
        \usepackage{fontspec}
        \usepackage{unicode-math}
    \fi

    \usepackage{fancyvrb} % verbatim replacement that allows latex
    \usepackage{grffile} % extends the file name processing of package graphics
                         % to support a larger range
    \makeatletter % fix for old versions of grffile with XeLaTeX
    \@ifpackagelater{grffile}{2019/11/01}
    {
      % Do nothing on new versions
    }
    {
      \def\Gread@@xetex#1{%
        \IfFileExists{"\Gin@base".bb}%
        {\Gread@eps{\Gin@base.bb}}%
        {\Gread@@xetex@aux#1}%
      }
    }
    \makeatother
    \usepackage[Export]{adjustbox} % Used to constrain images to a maximum size
    \adjustboxset{max size={0.9\linewidth}{0.9\paperheight}}

    % The hyperref package gives us a pdf with properly built
    % internal navigation ('pdf bookmarks' for the table of contents,
    % internal cross-reference links, web links for URLs, etc.)
    \usepackage{hyperref}
    % The default LaTeX title has an obnoxious amount of whitespace. By default,
    % titling removes some of it. It also provides customization options.
    \usepackage{titling}
    \usepackage{longtable} % longtable support required by pandoc >1.10
    \usepackage{booktabs}  % table support for pandoc > 1.12.2
    \usepackage{array}     % table support for pandoc >= 2.11.3
    \usepackage{calc}      % table minipage width calculation for pandoc >= 2.11.1
    \usepackage[inline]{enumitem} % IRkernel/repr support (it uses the enumerate* environment)
    \usepackage[normalem]{ulem} % ulem is needed to support strikethroughs (\sout)
                                % normalem makes italics be italics, not underlines
    \usepackage{mathrsfs}
    

    
    % Colors for the hyperref package
    \definecolor{urlcolor}{rgb}{0,.145,.698}
    \definecolor{linkcolor}{rgb}{.71,0.21,0.01}
    \definecolor{citecolor}{rgb}{.12,.54,.11}

    % ANSI colors
    \definecolor{ansi-black}{HTML}{3E424D}
    \definecolor{ansi-black-intense}{HTML}{282C36}
    \definecolor{ansi-red}{HTML}{E75C58}
    \definecolor{ansi-red-intense}{HTML}{B22B31}
    \definecolor{ansi-green}{HTML}{00A250}
    \definecolor{ansi-green-intense}{HTML}{007427}
    \definecolor{ansi-yellow}{HTML}{DDB62B}
    \definecolor{ansi-yellow-intense}{HTML}{B27D12}
    \definecolor{ansi-blue}{HTML}{208FFB}
    \definecolor{ansi-blue-intense}{HTML}{0065CA}
    \definecolor{ansi-magenta}{HTML}{D160C4}
    \definecolor{ansi-magenta-intense}{HTML}{A03196}
    \definecolor{ansi-cyan}{HTML}{60C6C8}
    \definecolor{ansi-cyan-intense}{HTML}{258F8F}
    \definecolor{ansi-white}{HTML}{C5C1B4}
    \definecolor{ansi-white-intense}{HTML}{A1A6B2}
    \definecolor{ansi-default-inverse-fg}{HTML}{FFFFFF}
    \definecolor{ansi-default-inverse-bg}{HTML}{000000}

    % common color for the border for error outputs.
    \definecolor{outerrorbackground}{HTML}{FFDFDF}

    % commands and environments needed by pandoc snippets
    % extracted from the output of `pandoc -s`
    \providecommand{\tightlist}{%
      \setlength{\itemsep}{0pt}\setlength{\parskip}{0pt}}
    \DefineVerbatimEnvironment{Highlighting}{Verbatim}{commandchars=\\\{\}}
    % Add ',fontsize=\small' for more characters per line
    \newenvironment{Shaded}{}{}
    \newcommand{\KeywordTok}[1]{\textcolor[rgb]{0.00,0.44,0.13}{\textbf{{#1}}}}
    \newcommand{\DataTypeTok}[1]{\textcolor[rgb]{0.56,0.13,0.00}{{#1}}}
    \newcommand{\DecValTok}[1]{\textcolor[rgb]{0.25,0.63,0.44}{{#1}}}
    \newcommand{\BaseNTok}[1]{\textcolor[rgb]{0.25,0.63,0.44}{{#1}}}
    \newcommand{\FloatTok}[1]{\textcolor[rgb]{0.25,0.63,0.44}{{#1}}}
    \newcommand{\CharTok}[1]{\textcolor[rgb]{0.25,0.44,0.63}{{#1}}}
    \newcommand{\StringTok}[1]{\textcolor[rgb]{0.25,0.44,0.63}{{#1}}}
    \newcommand{\CommentTok}[1]{\textcolor[rgb]{0.38,0.63,0.69}{\textit{{#1}}}}
    \newcommand{\OtherTok}[1]{\textcolor[rgb]{0.00,0.44,0.13}{{#1}}}
    \newcommand{\AlertTok}[1]{\textcolor[rgb]{1.00,0.00,0.00}{\textbf{{#1}}}}
    \newcommand{\FunctionTok}[1]{\textcolor[rgb]{0.02,0.16,0.49}{{#1}}}
    \newcommand{\RegionMarkerTok}[1]{{#1}}
    \newcommand{\ErrorTok}[1]{\textcolor[rgb]{1.00,0.00,0.00}{\textbf{{#1}}}}
    \newcommand{\NormalTok}[1]{{#1}}

    % Additional commands for more recent versions of Pandoc
    \newcommand{\ConstantTok}[1]{\textcolor[rgb]{0.53,0.00,0.00}{{#1}}}
    \newcommand{\SpecialCharTok}[1]{\textcolor[rgb]{0.25,0.44,0.63}{{#1}}}
    \newcommand{\VerbatimStringTok}[1]{\textcolor[rgb]{0.25,0.44,0.63}{{#1}}}
    \newcommand{\SpecialStringTok}[1]{\textcolor[rgb]{0.73,0.40,0.53}{{#1}}}
    \newcommand{\ImportTok}[1]{{#1}}
    \newcommand{\DocumentationTok}[1]{\textcolor[rgb]{0.73,0.13,0.13}{\textit{{#1}}}}
    \newcommand{\AnnotationTok}[1]{\textcolor[rgb]{0.38,0.63,0.69}{\textbf{\textit{{#1}}}}}
    \newcommand{\CommentVarTok}[1]{\textcolor[rgb]{0.38,0.63,0.69}{\textbf{\textit{{#1}}}}}
    \newcommand{\VariableTok}[1]{\textcolor[rgb]{0.10,0.09,0.49}{{#1}}}
    \newcommand{\ControlFlowTok}[1]{\textcolor[rgb]{0.00,0.44,0.13}{\textbf{{#1}}}}
    \newcommand{\OperatorTok}[1]{\textcolor[rgb]{0.40,0.40,0.40}{{#1}}}
    \newcommand{\BuiltInTok}[1]{{#1}}
    \newcommand{\ExtensionTok}[1]{{#1}}
    \newcommand{\PreprocessorTok}[1]{\textcolor[rgb]{0.74,0.48,0.00}{{#1}}}
    \newcommand{\AttributeTok}[1]{\textcolor[rgb]{0.49,0.56,0.16}{{#1}}}
    \newcommand{\InformationTok}[1]{\textcolor[rgb]{0.38,0.63,0.69}{\textbf{\textit{{#1}}}}}
    \newcommand{\WarningTok}[1]{\textcolor[rgb]{0.38,0.63,0.69}{\textbf{\textit{{#1}}}}}


    % Define a nice break command that doesn't care if a line doesn't already
    % exist.
    \def\br{\hspace*{\fill} \\* }
    % Math Jax compatibility definitions
    \def\gt{>}
    \def\lt{<}
    \let\Oldtex\TeX
    \let\Oldlatex\LaTeX
    \renewcommand{\TeX}{\textrm{\Oldtex}}
    \renewcommand{\LaTeX}{\textrm{\Oldlatex}}
    % Document parameters
    % Document title
    \title{hw04}
    
    
    
    
    
% Pygments definitions
\makeatletter
\def\PY@reset{\let\PY@it=\relax \let\PY@bf=\relax%
    \let\PY@ul=\relax \let\PY@tc=\relax%
    \let\PY@bc=\relax \let\PY@ff=\relax}
\def\PY@tok#1{\csname PY@tok@#1\endcsname}
\def\PY@toks#1+{\ifx\relax#1\empty\else%
    \PY@tok{#1}\expandafter\PY@toks\fi}
\def\PY@do#1{\PY@bc{\PY@tc{\PY@ul{%
    \PY@it{\PY@bf{\PY@ff{#1}}}}}}}
\def\PY#1#2{\PY@reset\PY@toks#1+\relax+\PY@do{#2}}

\@namedef{PY@tok@w}{\def\PY@tc##1{\textcolor[rgb]{0.73,0.73,0.73}{##1}}}
\@namedef{PY@tok@c}{\let\PY@it=\textit\def\PY@tc##1{\textcolor[rgb]{0.24,0.48,0.48}{##1}}}
\@namedef{PY@tok@cp}{\def\PY@tc##1{\textcolor[rgb]{0.61,0.40,0.00}{##1}}}
\@namedef{PY@tok@k}{\let\PY@bf=\textbf\def\PY@tc##1{\textcolor[rgb]{0.00,0.50,0.00}{##1}}}
\@namedef{PY@tok@kp}{\def\PY@tc##1{\textcolor[rgb]{0.00,0.50,0.00}{##1}}}
\@namedef{PY@tok@kt}{\def\PY@tc##1{\textcolor[rgb]{0.69,0.00,0.25}{##1}}}
\@namedef{PY@tok@o}{\def\PY@tc##1{\textcolor[rgb]{0.40,0.40,0.40}{##1}}}
\@namedef{PY@tok@ow}{\let\PY@bf=\textbf\def\PY@tc##1{\textcolor[rgb]{0.67,0.13,1.00}{##1}}}
\@namedef{PY@tok@nb}{\def\PY@tc##1{\textcolor[rgb]{0.00,0.50,0.00}{##1}}}
\@namedef{PY@tok@nf}{\def\PY@tc##1{\textcolor[rgb]{0.00,0.00,1.00}{##1}}}
\@namedef{PY@tok@nc}{\let\PY@bf=\textbf\def\PY@tc##1{\textcolor[rgb]{0.00,0.00,1.00}{##1}}}
\@namedef{PY@tok@nn}{\let\PY@bf=\textbf\def\PY@tc##1{\textcolor[rgb]{0.00,0.00,1.00}{##1}}}
\@namedef{PY@tok@ne}{\let\PY@bf=\textbf\def\PY@tc##1{\textcolor[rgb]{0.80,0.25,0.22}{##1}}}
\@namedef{PY@tok@nv}{\def\PY@tc##1{\textcolor[rgb]{0.10,0.09,0.49}{##1}}}
\@namedef{PY@tok@no}{\def\PY@tc##1{\textcolor[rgb]{0.53,0.00,0.00}{##1}}}
\@namedef{PY@tok@nl}{\def\PY@tc##1{\textcolor[rgb]{0.46,0.46,0.00}{##1}}}
\@namedef{PY@tok@ni}{\let\PY@bf=\textbf\def\PY@tc##1{\textcolor[rgb]{0.44,0.44,0.44}{##1}}}
\@namedef{PY@tok@na}{\def\PY@tc##1{\textcolor[rgb]{0.41,0.47,0.13}{##1}}}
\@namedef{PY@tok@nt}{\let\PY@bf=\textbf\def\PY@tc##1{\textcolor[rgb]{0.00,0.50,0.00}{##1}}}
\@namedef{PY@tok@nd}{\def\PY@tc##1{\textcolor[rgb]{0.67,0.13,1.00}{##1}}}
\@namedef{PY@tok@s}{\def\PY@tc##1{\textcolor[rgb]{0.73,0.13,0.13}{##1}}}
\@namedef{PY@tok@sd}{\let\PY@it=\textit\def\PY@tc##1{\textcolor[rgb]{0.73,0.13,0.13}{##1}}}
\@namedef{PY@tok@si}{\let\PY@bf=\textbf\def\PY@tc##1{\textcolor[rgb]{0.64,0.35,0.47}{##1}}}
\@namedef{PY@tok@se}{\let\PY@bf=\textbf\def\PY@tc##1{\textcolor[rgb]{0.67,0.36,0.12}{##1}}}
\@namedef{PY@tok@sr}{\def\PY@tc##1{\textcolor[rgb]{0.64,0.35,0.47}{##1}}}
\@namedef{PY@tok@ss}{\def\PY@tc##1{\textcolor[rgb]{0.10,0.09,0.49}{##1}}}
\@namedef{PY@tok@sx}{\def\PY@tc##1{\textcolor[rgb]{0.00,0.50,0.00}{##1}}}
\@namedef{PY@tok@m}{\def\PY@tc##1{\textcolor[rgb]{0.40,0.40,0.40}{##1}}}
\@namedef{PY@tok@gh}{\let\PY@bf=\textbf\def\PY@tc##1{\textcolor[rgb]{0.00,0.00,0.50}{##1}}}
\@namedef{PY@tok@gu}{\let\PY@bf=\textbf\def\PY@tc##1{\textcolor[rgb]{0.50,0.00,0.50}{##1}}}
\@namedef{PY@tok@gd}{\def\PY@tc##1{\textcolor[rgb]{0.63,0.00,0.00}{##1}}}
\@namedef{PY@tok@gi}{\def\PY@tc##1{\textcolor[rgb]{0.00,0.52,0.00}{##1}}}
\@namedef{PY@tok@gr}{\def\PY@tc##1{\textcolor[rgb]{0.89,0.00,0.00}{##1}}}
\@namedef{PY@tok@ge}{\let\PY@it=\textit}
\@namedef{PY@tok@gs}{\let\PY@bf=\textbf}
\@namedef{PY@tok@gp}{\let\PY@bf=\textbf\def\PY@tc##1{\textcolor[rgb]{0.00,0.00,0.50}{##1}}}
\@namedef{PY@tok@go}{\def\PY@tc##1{\textcolor[rgb]{0.44,0.44,0.44}{##1}}}
\@namedef{PY@tok@gt}{\def\PY@tc##1{\textcolor[rgb]{0.00,0.27,0.87}{##1}}}
\@namedef{PY@tok@err}{\def\PY@bc##1{{\setlength{\fboxsep}{\string -\fboxrule}\fcolorbox[rgb]{1.00,0.00,0.00}{1,1,1}{\strut ##1}}}}
\@namedef{PY@tok@kc}{\let\PY@bf=\textbf\def\PY@tc##1{\textcolor[rgb]{0.00,0.50,0.00}{##1}}}
\@namedef{PY@tok@kd}{\let\PY@bf=\textbf\def\PY@tc##1{\textcolor[rgb]{0.00,0.50,0.00}{##1}}}
\@namedef{PY@tok@kn}{\let\PY@bf=\textbf\def\PY@tc##1{\textcolor[rgb]{0.00,0.50,0.00}{##1}}}
\@namedef{PY@tok@kr}{\let\PY@bf=\textbf\def\PY@tc##1{\textcolor[rgb]{0.00,0.50,0.00}{##1}}}
\@namedef{PY@tok@bp}{\def\PY@tc##1{\textcolor[rgb]{0.00,0.50,0.00}{##1}}}
\@namedef{PY@tok@fm}{\def\PY@tc##1{\textcolor[rgb]{0.00,0.00,1.00}{##1}}}
\@namedef{PY@tok@vc}{\def\PY@tc##1{\textcolor[rgb]{0.10,0.09,0.49}{##1}}}
\@namedef{PY@tok@vg}{\def\PY@tc##1{\textcolor[rgb]{0.10,0.09,0.49}{##1}}}
\@namedef{PY@tok@vi}{\def\PY@tc##1{\textcolor[rgb]{0.10,0.09,0.49}{##1}}}
\@namedef{PY@tok@vm}{\def\PY@tc##1{\textcolor[rgb]{0.10,0.09,0.49}{##1}}}
\@namedef{PY@tok@sa}{\def\PY@tc##1{\textcolor[rgb]{0.73,0.13,0.13}{##1}}}
\@namedef{PY@tok@sb}{\def\PY@tc##1{\textcolor[rgb]{0.73,0.13,0.13}{##1}}}
\@namedef{PY@tok@sc}{\def\PY@tc##1{\textcolor[rgb]{0.73,0.13,0.13}{##1}}}
\@namedef{PY@tok@dl}{\def\PY@tc##1{\textcolor[rgb]{0.73,0.13,0.13}{##1}}}
\@namedef{PY@tok@s2}{\def\PY@tc##1{\textcolor[rgb]{0.73,0.13,0.13}{##1}}}
\@namedef{PY@tok@sh}{\def\PY@tc##1{\textcolor[rgb]{0.73,0.13,0.13}{##1}}}
\@namedef{PY@tok@s1}{\def\PY@tc##1{\textcolor[rgb]{0.73,0.13,0.13}{##1}}}
\@namedef{PY@tok@mb}{\def\PY@tc##1{\textcolor[rgb]{0.40,0.40,0.40}{##1}}}
\@namedef{PY@tok@mf}{\def\PY@tc##1{\textcolor[rgb]{0.40,0.40,0.40}{##1}}}
\@namedef{PY@tok@mh}{\def\PY@tc##1{\textcolor[rgb]{0.40,0.40,0.40}{##1}}}
\@namedef{PY@tok@mi}{\def\PY@tc##1{\textcolor[rgb]{0.40,0.40,0.40}{##1}}}
\@namedef{PY@tok@il}{\def\PY@tc##1{\textcolor[rgb]{0.40,0.40,0.40}{##1}}}
\@namedef{PY@tok@mo}{\def\PY@tc##1{\textcolor[rgb]{0.40,0.40,0.40}{##1}}}
\@namedef{PY@tok@ch}{\let\PY@it=\textit\def\PY@tc##1{\textcolor[rgb]{0.24,0.48,0.48}{##1}}}
\@namedef{PY@tok@cm}{\let\PY@it=\textit\def\PY@tc##1{\textcolor[rgb]{0.24,0.48,0.48}{##1}}}
\@namedef{PY@tok@cpf}{\let\PY@it=\textit\def\PY@tc##1{\textcolor[rgb]{0.24,0.48,0.48}{##1}}}
\@namedef{PY@tok@c1}{\let\PY@it=\textit\def\PY@tc##1{\textcolor[rgb]{0.24,0.48,0.48}{##1}}}
\@namedef{PY@tok@cs}{\let\PY@it=\textit\def\PY@tc##1{\textcolor[rgb]{0.24,0.48,0.48}{##1}}}

\def\PYZbs{\char`\\}
\def\PYZus{\char`\_}
\def\PYZob{\char`\{}
\def\PYZcb{\char`\}}
\def\PYZca{\char`\^}
\def\PYZam{\char`\&}
\def\PYZlt{\char`\<}
\def\PYZgt{\char`\>}
\def\PYZsh{\char`\#}
\def\PYZpc{\char`\%}
\def\PYZdl{\char`\$}
\def\PYZhy{\char`\-}
\def\PYZsq{\char`\'}
\def\PYZdq{\char`\"}
\def\PYZti{\char`\~}
% for compatibility with earlier versions
\def\PYZat{@}
\def\PYZlb{[}
\def\PYZrb{]}
\makeatother


    % For linebreaks inside Verbatim environment from package fancyvrb.
    \makeatletter
        \newbox\Wrappedcontinuationbox
        \newbox\Wrappedvisiblespacebox
        \newcommand*\Wrappedvisiblespace {\textcolor{red}{\textvisiblespace}}
        \newcommand*\Wrappedcontinuationsymbol {\textcolor{red}{\llap{\tiny$\m@th\hookrightarrow$}}}
        \newcommand*\Wrappedcontinuationindent {3ex }
        \newcommand*\Wrappedafterbreak {\kern\Wrappedcontinuationindent\copy\Wrappedcontinuationbox}
        % Take advantage of the already applied Pygments mark-up to insert
        % potential linebreaks for TeX processing.
        %        {, <, #, %, $, ' and ": go to next line.
        %        _, }, ^, &, >, - and ~: stay at end of broken line.
        % Use of \textquotesingle for straight quote.
        \newcommand*\Wrappedbreaksatspecials {%
            \def\PYGZus{\discretionary{\char`\_}{\Wrappedafterbreak}{\char`\_}}%
            \def\PYGZob{\discretionary{}{\Wrappedafterbreak\char`\{}{\char`\{}}%
            \def\PYGZcb{\discretionary{\char`\}}{\Wrappedafterbreak}{\char`\}}}%
            \def\PYGZca{\discretionary{\char`\^}{\Wrappedafterbreak}{\char`\^}}%
            \def\PYGZam{\discretionary{\char`\&}{\Wrappedafterbreak}{\char`\&}}%
            \def\PYGZlt{\discretionary{}{\Wrappedafterbreak\char`\<}{\char`\<}}%
            \def\PYGZgt{\discretionary{\char`\>}{\Wrappedafterbreak}{\char`\>}}%
            \def\PYGZsh{\discretionary{}{\Wrappedafterbreak\char`\#}{\char`\#}}%
            \def\PYGZpc{\discretionary{}{\Wrappedafterbreak\char`\%}{\char`\%}}%
            \def\PYGZdl{\discretionary{}{\Wrappedafterbreak\char`\$}{\char`\$}}%
            \def\PYGZhy{\discretionary{\char`\-}{\Wrappedafterbreak}{\char`\-}}%
            \def\PYGZsq{\discretionary{}{\Wrappedafterbreak\textquotesingle}{\textquotesingle}}%
            \def\PYGZdq{\discretionary{}{\Wrappedafterbreak\char`\"}{\char`\"}}%
            \def\PYGZti{\discretionary{\char`\~}{\Wrappedafterbreak}{\char`\~}}%
        }
        % Some characters . , ; ? ! / are not pygmentized.
        % This macro makes them "active" and they will insert potential linebreaks
        \newcommand*\Wrappedbreaksatpunct {%
            \lccode`\~`\.\lowercase{\def~}{\discretionary{\hbox{\char`\.}}{\Wrappedafterbreak}{\hbox{\char`\.}}}%
            \lccode`\~`\,\lowercase{\def~}{\discretionary{\hbox{\char`\,}}{\Wrappedafterbreak}{\hbox{\char`\,}}}%
            \lccode`\~`\;\lowercase{\def~}{\discretionary{\hbox{\char`\;}}{\Wrappedafterbreak}{\hbox{\char`\;}}}%
            \lccode`\~`\:\lowercase{\def~}{\discretionary{\hbox{\char`\:}}{\Wrappedafterbreak}{\hbox{\char`\:}}}%
            \lccode`\~`\?\lowercase{\def~}{\discretionary{\hbox{\char`\?}}{\Wrappedafterbreak}{\hbox{\char`\?}}}%
            \lccode`\~`\!\lowercase{\def~}{\discretionary{\hbox{\char`\!}}{\Wrappedafterbreak}{\hbox{\char`\!}}}%
            \lccode`\~`\/\lowercase{\def~}{\discretionary{\hbox{\char`\/}}{\Wrappedafterbreak}{\hbox{\char`\/}}}%
            \catcode`\.\active
            \catcode`\,\active
            \catcode`\;\active
            \catcode`\:\active
            \catcode`\?\active
            \catcode`\!\active
            \catcode`\/\active
            \lccode`\~`\~
        }
    \makeatother

    \let\OriginalVerbatim=\Verbatim
    \makeatletter
    \renewcommand{\Verbatim}[1][1]{%
        %\parskip\z@skip
        \sbox\Wrappedcontinuationbox {\Wrappedcontinuationsymbol}%
        \sbox\Wrappedvisiblespacebox {\FV@SetupFont\Wrappedvisiblespace}%
        \def\FancyVerbFormatLine ##1{\hsize\linewidth
            \vtop{\raggedright\hyphenpenalty\z@\exhyphenpenalty\z@
                \doublehyphendemerits\z@\finalhyphendemerits\z@
                \strut ##1\strut}%
        }%
        % If the linebreak is at a space, the latter will be displayed as visible
        % space at end of first line, and a continuation symbol starts next line.
        % Stretch/shrink are however usually zero for typewriter font.
        \def\FV@Space {%
            \nobreak\hskip\z@ plus\fontdimen3\font minus\fontdimen4\font
            \discretionary{\copy\Wrappedvisiblespacebox}{\Wrappedafterbreak}
            {\kern\fontdimen2\font}%
        }%

        % Allow breaks at special characters using \PYG... macros.
        \Wrappedbreaksatspecials
        % Breaks at punctuation characters . , ; ? ! and / need catcode=\active
        \OriginalVerbatim[#1,codes*=\Wrappedbreaksatpunct]%
    }
    \makeatother

    % Exact colors from NB
    \definecolor{incolor}{HTML}{303F9F}
    \definecolor{outcolor}{HTML}{D84315}
    \definecolor{cellborder}{HTML}{CFCFCF}
    \definecolor{cellbackground}{HTML}{F7F7F7}

    % prompt
    \makeatletter
    \newcommand{\boxspacing}{\kern\kvtcb@left@rule\kern\kvtcb@boxsep}
    \makeatother
    \newcommand{\prompt}[4]{
        {\ttfamily\llap{{\color{#2}[#3]:\hspace{3pt}#4}}\vspace{-\baselineskip}}
    }
    

    
    % Prevent overflowing lines due to hard-to-break entities
    \sloppy
    % Setup hyperref package
    \hypersetup{
      breaklinks=true,  % so long urls are correctly broken across lines
      colorlinks=true,
      urlcolor=urlcolor,
      linkcolor=linkcolor,
      citecolor=citecolor,
      }
    % Slightly bigger margins than the latex defaults
    
    \geometry{verbose,tmargin=1in,bmargin=1in,lmargin=1in,rmargin=1in}
    
    

\begin{document}
    
    \maketitle
    
    

    
    \hypertarget{metadata}{%
\section{Metadata}\label{metadata}}

\begin{Shaded}
\begin{Highlighting}[]
\FunctionTok{Course}\KeywordTok{:}\AttributeTok{  DS 5100}
\FunctionTok{Module}\KeywordTok{:}\AttributeTok{  04 Functions HW}
\FunctionTok{Title}\KeywordTok{:}\AttributeTok{   Fighting Forest Fires with Functions}
\FunctionTok{Author}\KeywordTok{:}\AttributeTok{  R.C. Alvarado (adapted)}
\FunctionTok{Datae}\KeywordTok{:}\AttributeTok{   7 July 2023}
\end{Highlighting}
\end{Shaded}

    \hypertarget{student-info}{%
\section{Student Info}\label{student-info}}

\begin{itemize}
\tightlist
\item
  Name: Lindley Slipetz
\item
  Net ID: ddj6tu
\item
  URL of this file in GitHub:
  https://github.com/sliplr19/DS5100-ddj6tu/tree/main/lessons/M04
\end{itemize}

    \hypertarget{instructions}{%
\section{Instructions}\label{instructions}}

In your \textbf{private course repo on Rivanna}, write a Jupyter
notebook running Python that performs the numbered tasks below.

For each task, create one or more code cells to perform the task.

Save your notebook in the \texttt{M04} directory as \texttt{hw04.ipynb}.

Add and commit these files to your repo.

Then push your commits to your repo on GitHib.

Be sure to fill out the \textbf{Student Info} block above.

To submit your homework, save the notebook as a PDF and upload it to
GradeScope, following the instructions.

\textbf{TOTAL POINTS: 14}

    \hypertarget{overview}{%
\section{Overview}\label{overview}}

In this homework, you will work with the
\href{https://archive.ics.uci.edu/ml/datasets/Forest+Fires}{Forest Fires
Data Set from UCI}.

There is a local copy of these data as a CSV file in the \texttt{HW}
directory for this module in the course repo.

You will create a group of related functions to process these data.

This notebook will set the table for you by importing and structuring
the data first.

    \hypertarget{setting-up}{%
\section{Setting Up}\label{setting-up}}

First, we read in our local copy of the dataset and save it as a list of
lines.

    \begin{tcolorbox}[breakable, size=fbox, boxrule=1pt, pad at break*=1mm,colback=cellbackground, colframe=cellborder]
\prompt{In}{incolor}{1}{\boxspacing}
\begin{Verbatim}[commandchars=\\\{\}]
\PY{n}{data\PYZus{}file} \PY{o}{=} \PY{n+nb}{open}\PY{p}{(}\PY{l+s+s1}{\PYZsq{}}\PY{l+s+s1}{uci\PYZus{}mldb\PYZus{}forestfires.csv}\PY{l+s+s1}{\PYZsq{}}\PY{p}{,} \PY{l+s+s1}{\PYZsq{}}\PY{l+s+s1}{r}\PY{l+s+s1}{\PYZsq{}}\PY{p}{)}\PY{o}{.}\PY{n}{readlines}\PY{p}{(}\PY{p}{)}
\end{Verbatim}
\end{tcolorbox}

    Then, we inspect first ten lines, replacing commas with tabs for
readability.

    \begin{tcolorbox}[breakable, size=fbox, boxrule=1pt, pad at break*=1mm,colback=cellbackground, colframe=cellborder]
\prompt{In}{incolor}{2}{\boxspacing}
\begin{Verbatim}[commandchars=\\\{\}]
\PY{k}{for} \PY{n}{row} \PY{o+ow}{in} \PY{n}{data\PYZus{}file}\PY{p}{[}\PY{p}{:}\PY{l+m+mi}{10}\PY{p}{]}\PY{p}{:}
    \PY{n}{row} \PY{o}{=} \PY{n}{row}\PY{o}{.}\PY{n}{replace}\PY{p}{(}\PY{l+s+s1}{\PYZsq{}}\PY{l+s+s1}{,}\PY{l+s+s1}{\PYZsq{}}\PY{p}{,} \PY{l+s+s1}{\PYZsq{}}\PY{l+s+se}{\PYZbs{}t}\PY{l+s+s1}{\PYZsq{}}\PY{p}{)}
    \PY{n+nb}{print}\PY{p}{(}\PY{n}{row}\PY{p}{,} \PY{n}{end}\PY{o}{=}\PY{l+s+s1}{\PYZsq{}}\PY{l+s+s1}{\PYZsq{}}\PY{p}{)}
\end{Verbatim}
\end{tcolorbox}

    \begin{Verbatim}[commandchars=\\\{\}]
X       Y       month   day     FFMC    DMC     DC      ISI     temp    RH
wind    rain    area
7       5       mar     fri     86.2    26.2    94.3    5.1     8.2     51
6.7     0.0     0.0
7       4       oct     tue     90.6    35.4    669.1   6.7     18.0    33
0.9     0.0     0.0
7       4       oct     sat     90.6    43.7    686.9   6.7     14.6    33
1.3     0.0     0.0
8       6       mar     fri     91.7    33.3    77.5    9.0     8.3     97
4.0     0.2     0.0
8       6       mar     sun     89.3    51.3    102.2   9.6     11.4    99
1.8     0.0     0.0
8       6       aug     sun     92.3    85.3    488.0   14.7    22.2    29
5.4     0.0     0.0
8       6       aug     mon     92.3    88.9    495.6   8.5     24.1    27
3.1     0.0     0.0
8       6       aug     mon     91.5    145.4   608.2   10.7    8.0     86
2.2     0.0     0.0
8       6       sep     tue     91.0    129.5   692.6   7.0     13.1    63
5.4     0.0     0.0
    \end{Verbatim}

    \hypertarget{convert-csv-into-datafame-like-data-structure}{%
\subsection{Convert CSV into Datafame-like Data
Structure}\label{convert-csv-into-datafame-like-data-structure}}

    We use a helper function to convert the data into the form of a
dataframe-like dictionary.

That is, we convert a list of rows into a dictionary of columns, each
cast to the appropriate data type.

Later, we will use Pandas and R dataframes to do this work.

    First, we define the data types by inspecting the data and creating a
dictionary of lambda functions to do our casting.

    \begin{tcolorbox}[breakable, size=fbox, boxrule=1pt, pad at break*=1mm,colback=cellbackground, colframe=cellborder]
\prompt{In}{incolor}{3}{\boxspacing}
\begin{Verbatim}[commandchars=\\\{\}]
\PY{n}{dtypes} \PY{o}{=} \PY{p}{[}\PY{l+s+s1}{\PYZsq{}}\PY{l+s+s1}{i}\PY{l+s+s1}{\PYZsq{}}\PY{p}{,} \PY{l+s+s1}{\PYZsq{}}\PY{l+s+s1}{i}\PY{l+s+s1}{\PYZsq{}}\PY{p}{,} \PY{l+s+s1}{\PYZsq{}}\PY{l+s+s1}{s}\PY{l+s+s1}{\PYZsq{}}\PY{p}{,} \PY{l+s+s1}{\PYZsq{}}\PY{l+s+s1}{s}\PY{l+s+s1}{\PYZsq{}}\PY{p}{,} \PY{l+s+s1}{\PYZsq{}}\PY{l+s+s1}{f}\PY{l+s+s1}{\PYZsq{}}\PY{p}{,} \PY{l+s+s1}{\PYZsq{}}\PY{l+s+s1}{f}\PY{l+s+s1}{\PYZsq{}}\PY{p}{,} \PY{l+s+s1}{\PYZsq{}}\PY{l+s+s1}{f}\PY{l+s+s1}{\PYZsq{}}\PY{p}{,} \PY{l+s+s1}{\PYZsq{}}\PY{l+s+s1}{f}\PY{l+s+s1}{\PYZsq{}}\PY{p}{,} \PY{l+s+s1}{\PYZsq{}}\PY{l+s+s1}{f}\PY{l+s+s1}{\PYZsq{}}\PY{p}{,} \PY{l+s+s1}{\PYZsq{}}\PY{l+s+s1}{i}\PY{l+s+s1}{\PYZsq{}}\PY{p}{,} \PY{l+s+s1}{\PYZsq{}}\PY{l+s+s1}{f}\PY{l+s+s1}{\PYZsq{}}\PY{p}{,} \PY{l+s+s1}{\PYZsq{}}\PY{l+s+s1}{f}\PY{l+s+s1}{\PYZsq{}}\PY{p}{,} \PY{l+s+s1}{\PYZsq{}}\PY{l+s+s1}{f}\PY{l+s+s1}{\PYZsq{}}\PY{p}{]}
\PY{c+c1}{\PYZsh{} dtypes = list(\PYZdq{}iissfffffifff\PYZdq{}) \PYZsh{} We could have done it this way, too}

\PY{n}{caster} \PY{o}{=} \PY{p}{\PYZob{}}
    \PY{l+s+s1}{\PYZsq{}}\PY{l+s+s1}{i}\PY{l+s+s1}{\PYZsq{}}\PY{p}{:} \PY{k}{lambda} \PY{n}{x}\PY{p}{:} \PY{n+nb}{int}\PY{p}{(}\PY{n}{x}\PY{p}{)}\PY{p}{,}
    \PY{l+s+s1}{\PYZsq{}}\PY{l+s+s1}{s}\PY{l+s+s1}{\PYZsq{}}\PY{p}{:} \PY{k}{lambda} \PY{n}{x}\PY{p}{:} \PY{n+nb}{str}\PY{p}{(}\PY{n}{x}\PY{p}{)}\PY{p}{,}
    \PY{l+s+s1}{\PYZsq{}}\PY{l+s+s1}{f}\PY{l+s+s1}{\PYZsq{}}\PY{p}{:} \PY{k}{lambda} \PY{n}{x}\PY{p}{:} \PY{n+nb}{float}\PY{p}{(}\PY{n}{x}\PY{p}{)}
\PY{p}{\PYZcb{}}
\end{Verbatim}
\end{tcolorbox}

    Next, we grab the column names from the first row or list.

Note that \texttt{.strip()} is a string function that removes extra
whitespace from before and after a string.

    \begin{tcolorbox}[breakable, size=fbox, boxrule=1pt, pad at break*=1mm,colback=cellbackground, colframe=cellborder]
\prompt{In}{incolor}{4}{\boxspacing}
\begin{Verbatim}[commandchars=\\\{\}]
\PY{n}{cols} \PY{o}{=} \PY{n}{data\PYZus{}file}\PY{p}{[}\PY{l+m+mi}{0}\PY{p}{]}\PY{o}{.}\PY{n}{strip}\PY{p}{(}\PY{p}{)}\PY{o}{.}\PY{n}{split}\PY{p}{(}\PY{l+s+s1}{\PYZsq{}}\PY{l+s+s1}{,}\PY{l+s+s1}{\PYZsq{}}\PY{p}{)}
\end{Verbatim}
\end{tcolorbox}

    Finally, we iterate through the list of rows and flip them into a
dictionary of columns.

The key of each dictionary element is the columns name, and the value is
a list of values with a common data type.

    \begin{tcolorbox}[breakable, size=fbox, boxrule=1pt, pad at break*=1mm,colback=cellbackground, colframe=cellborder]
\prompt{In}{incolor}{5}{\boxspacing}
\begin{Verbatim}[commandchars=\\\{\}]
\PY{c+c1}{\PYZsh{} Get the rows, but not the first, and convert them into lists}
\PY{n}{rows} \PY{o}{=} \PY{p}{[}\PY{n}{line}\PY{o}{.}\PY{n}{strip}\PY{p}{(}\PY{p}{)}\PY{o}{.}\PY{n}{split}\PY{p}{(}\PY{l+s+s1}{\PYZsq{}}\PY{l+s+s1}{,}\PY{l+s+s1}{\PYZsq{}}\PY{p}{)} \PY{k}{for} \PY{n}{line} \PY{o+ow}{in} \PY{n}{data\PYZus{}file}\PY{p}{[}\PY{l+m+mi}{1}\PY{p}{:}\PY{p}{]}\PY{p}{]}

\PY{c+c1}{\PYZsh{} Initialize the dataframe by defining a dictionary of lists, with each column name as a key}
\PY{n}{firedata} \PY{o}{=} \PY{p}{\PYZob{}}\PY{n}{col}\PY{p}{:}\PY{p}{[}\PY{p}{]} \PY{k}{for} \PY{n}{col} \PY{o+ow}{in} \PY{n}{cols}\PY{p}{\PYZcb{}}

\PY{c+c1}{\PYZsh{} Iterate through the rows and convert them to columns }
\PY{k}{for} \PY{n}{row} \PY{o+ow}{in} \PY{n}{rows}\PY{p}{:}
    \PY{k}{for} \PY{n}{j}\PY{p}{,} \PY{n}{col} \PY{o+ow}{in} \PY{n+nb}{enumerate}\PY{p}{(}\PY{n}{row}\PY{p}{)}\PY{p}{:}
        \PY{n}{firedata}\PY{p}{[}\PY{n}{cols}\PY{p}{[}\PY{n}{j}\PY{p}{]}\PY{p}{]}\PY{o}{.}\PY{n}{append}\PY{p}{(}\PY{n}{caster}\PY{p}{[}\PY{n}{dtypes}\PY{p}{[}\PY{n}{j}\PY{p}{]}\PY{p}{]}\PY{p}{(}\PY{n}{col}\PY{p}{)}\PY{p}{)}
\end{Verbatim}
\end{tcolorbox}

    Test to see if it worked \ldots{}

    \begin{tcolorbox}[breakable, size=fbox, boxrule=1pt, pad at break*=1mm,colback=cellbackground, colframe=cellborder]
\prompt{In}{incolor}{6}{\boxspacing}
\begin{Verbatim}[commandchars=\\\{\}]
\PY{n}{firedata}\PY{p}{[}\PY{l+s+s1}{\PYZsq{}}\PY{l+s+s1}{Y}\PY{l+s+s1}{\PYZsq{}}\PY{p}{]}\PY{p}{[}\PY{p}{:}\PY{l+m+mi}{5}\PY{p}{]}
\end{Verbatim}
\end{tcolorbox}

            \begin{tcolorbox}[breakable, size=fbox, boxrule=.5pt, pad at break*=1mm, opacityfill=0]
\prompt{Out}{outcolor}{6}{\boxspacing}
\begin{Verbatim}[commandchars=\\\{\}]
[5, 4, 4, 6, 6]
\end{Verbatim}
\end{tcolorbox}
        
    \hypertarget{working-with-spatial-coordinates-x-y}{%
\section{\texorpdfstring{Working with spatial coordinates \texttt{X},
\texttt{Y}}{Working with spatial coordinates X, Y}}\label{working-with-spatial-coordinates-x-y}}

For the first tasks, we grab the first two columns of our table, which
define the spatial coordinates within the Monteshino park map.

    \begin{tcolorbox}[breakable, size=fbox, boxrule=1pt, pad at break*=1mm,colback=cellbackground, colframe=cellborder]
\prompt{In}{incolor}{7}{\boxspacing}
\begin{Verbatim}[commandchars=\\\{\}]
\PY{n}{X}\PY{p}{,} \PY{n}{Y} \PY{o}{=} \PY{n}{firedata}\PY{p}{[}\PY{l+s+s1}{\PYZsq{}}\PY{l+s+s1}{X}\PY{l+s+s1}{\PYZsq{}}\PY{p}{]}\PY{p}{,} \PY{n}{firedata}\PY{p}{[}\PY{l+s+s1}{\PYZsq{}}\PY{l+s+s1}{Y}\PY{l+s+s1}{\PYZsq{}}\PY{p}{]}
\end{Verbatim}
\end{tcolorbox}

    \begin{tcolorbox}[breakable, size=fbox, boxrule=1pt, pad at break*=1mm,colback=cellbackground, colframe=cellborder]
\prompt{In}{incolor}{8}{\boxspacing}
\begin{Verbatim}[commandchars=\\\{\}]
\PY{n}{X}\PY{p}{[}\PY{p}{:}\PY{l+m+mi}{10}\PY{p}{]}\PY{p}{,} \PY{n}{Y}\PY{p}{[}\PY{p}{:}\PY{l+m+mi}{10}\PY{p}{]}
\end{Verbatim}
\end{tcolorbox}

            \begin{tcolorbox}[breakable, size=fbox, boxrule=.5pt, pad at break*=1mm, opacityfill=0]
\prompt{Out}{outcolor}{8}{\boxspacing}
\begin{Verbatim}[commandchars=\\\{\}]
([7, 7, 7, 8, 8, 8, 8, 8, 8, 7], [5, 4, 4, 6, 6, 6, 6, 6, 6, 5])
\end{Verbatim}
\end{tcolorbox}
        
    \hypertarget{task-1}{%
\subsection{Task 1}\label{task-1}}

(2 points)

Write a function called \texttt{coord\_builder()} with these
requirements:

\begin{itemize}
\tightlist
\item
  Takes two lists, X and Y, as inputs. X and Y must be of equal length.
\item
  Returns a list of tuples
  \texttt{{[}(x1,y1),\ (x2,y2),\ ...,\ (xn,yn){]}} where
  \texttt{(xi,yi)} are the ordered pairs from X and Y.
\item
  Uses the \texttt{zip()} function to create the returned list.
\item
  Use a list comprehension to actually build the returned list.
\item
  Contains a docstring with short description of the function.
\end{itemize}

    \begin{tcolorbox}[breakable, size=fbox, boxrule=1pt, pad at break*=1mm,colback=cellbackground, colframe=cellborder]
\prompt{In}{incolor}{28}{\boxspacing}
\begin{Verbatim}[commandchars=\\\{\}]
\PY{k}{def} \PY{n+nf}{coord\PYZus{}builder}\PY{p}{(}\PY{n}{X}\PY{p}{,}\PY{n}{Y}\PY{p}{)}\PY{p}{:}
    \PY{l+s+sd}{\PYZsq{}\PYZsq{}\PYZsq{}}
\PY{l+s+sd}{    This is the \PYZdq{}docstring\PYZdq{} of xoord\PYZus{}builder. }
\PY{l+s+sd}{    }
\PY{l+s+sd}{    PURPOSE: Given two lists, create a list of the tuples of the }
\PY{l+s+sd}{    members of the two listws.}
\PY{l+s+sd}{    }
\PY{l+s+sd}{    INPUTS}
\PY{l+s+sd}{    X   list }
\PY{l+s+sd}{    Y   list}
\PY{l+s+sd}{    }
\PY{l+s+sd}{    OUTPUT}
\PY{l+s+sd}{    list of tuples}
\PY{l+s+sd}{    \PYZsq{}\PYZsq{}\PYZsq{}}
    \PY{k}{if}\PY{p}{(}\PY{n+nb}{len}\PY{p}{(}\PY{n}{X}\PY{p}{)} \PY{o}{!=} \PY{n+nb}{len}\PY{p}{(}\PY{n}{Y}\PY{p}{)}\PY{p}{)}\PY{p}{:}
        \PY{n+nb}{print}\PY{p}{(}\PY{l+s+s2}{\PYZdq{}}\PY{l+s+s2}{Error: These lists are not the same length!}\PY{l+s+s2}{\PYZdq{}}\PY{p}{)}
    \PY{k}{else}\PY{p}{:}
        \PY{n}{zipxy} \PY{o}{=} \PY{n+nb}{zip}\PY{p}{(}\PY{n}{X}\PY{p}{,}\PY{n}{Y}\PY{p}{)}
        \PY{p}{[}\PY{n+nb}{print}\PY{p}{(}\PY{n+nb}{tuple}\PY{p}{(}\PY{n}{z}\PY{p}{)}\PY{p}{)} \PY{k}{for} \PY{n}{z} \PY{o+ow}{in} \PY{n}{zipxy}\PY{p}{]}
        
\end{Verbatim}
\end{tcolorbox}

    \hypertarget{task-2}{%
\subsection{Task 2}\label{task-2}}

(1 PT)

Call your \texttt{coord\_builder()} function, passing in \texttt{X} and
\texttt{Y}.

Then print the first ten tuples.

    \begin{tcolorbox}[breakable, size=fbox, boxrule=1pt, pad at break*=1mm,colback=cellbackground, colframe=cellborder]
\prompt{In}{incolor}{30}{\boxspacing}
\begin{Verbatim}[commandchars=\\\{\}]
\PY{n}{X10} \PY{o}{=} \PY{n}{X}\PY{p}{[}\PY{p}{:}\PY{l+m+mi}{10}\PY{p}{]}
\PY{n}{Y10} \PY{o}{=} \PY{n}{Y}\PY{p}{[}\PY{p}{:}\PY{l+m+mi}{10}\PY{p}{]}
\PY{n}{coord\PYZus{}builder}\PY{p}{(}\PY{n}{X10}\PY{p}{,} \PY{n}{Y10}\PY{p}{)}
\end{Verbatim}
\end{tcolorbox}

    \begin{Verbatim}[commandchars=\\\{\}]
(7, 5)
(7, 4)
(7, 4)
(8, 6)
(8, 6)
(8, 6)
(8, 6)
(8, 6)
(8, 6)
(7, 5)
    \end{Verbatim}

    \hypertarget{working-with-area}{%
\section{Working with AREA}\label{working-with-area}}

Next, we work the area column of our data.

    \begin{tcolorbox}[breakable, size=fbox, boxrule=1pt, pad at break*=1mm,colback=cellbackground, colframe=cellborder]
\prompt{In}{incolor}{47}{\boxspacing}
\begin{Verbatim}[commandchars=\\\{\}]
\PY{n}{area} \PY{o}{=} \PY{n}{firedata}\PY{p}{[}\PY{l+s+s1}{\PYZsq{}}\PY{l+s+s1}{area}\PY{l+s+s1}{\PYZsq{}}\PY{p}{]}
\end{Verbatim}
\end{tcolorbox}

    \begin{tcolorbox}[breakable, size=fbox, boxrule=1pt, pad at break*=1mm,colback=cellbackground, colframe=cellborder]
\prompt{In}{incolor}{32}{\boxspacing}
\begin{Verbatim}[commandchars=\\\{\}]
\PY{n}{area}\PY{p}{[}\PY{o}{\PYZhy{}}\PY{l+m+mi}{10}\PY{p}{:}\PY{p}{]}
\end{Verbatim}
\end{tcolorbox}

            \begin{tcolorbox}[breakable, size=fbox, boxrule=.5pt, pad at break*=1mm, opacityfill=0]
\prompt{Out}{outcolor}{32}{\boxspacing}
\begin{Verbatim}[commandchars=\\\{\}]
[0.0, 0.0, 2.17, 0.43, 0.0, 6.44, 54.29, 11.16, 0.0, 0.0]
\end{Verbatim}
\end{tcolorbox}
        
    \hypertarget{task-3}{%
\subsection{Task 3}\label{task-3}}

(1 PT)

Write code to print the minimum area and maximum area in a tuple
\texttt{(min\_value,\ max\_value)}.

Save \texttt{min\_value} and \texttt{max\_value} as floats.

    \begin{tcolorbox}[breakable, size=fbox, boxrule=1pt, pad at break*=1mm,colback=cellbackground, colframe=cellborder]
\prompt{In}{incolor}{34}{\boxspacing}
\begin{Verbatim}[commandchars=\\\{\}]
\PY{k}{def} \PY{n+nf}{min\PYZus{}max}\PY{p}{(}\PY{n}{area}\PY{p}{)}\PY{p}{:}
    \PY{n}{min\PYZus{}value} \PY{o}{=} \PY{n+nb}{float}\PY{p}{(}\PY{n+nb}{min}\PY{p}{(}\PY{n}{area}\PY{p}{)}\PY{p}{)}
    \PY{n}{max\PYZus{}value} \PY{o}{=} \PY{n+nb}{float}\PY{p}{(}\PY{n+nb}{max}\PY{p}{(}\PY{n}{area}\PY{p}{)}\PY{p}{)}
    \PY{n+nb}{print}\PY{p}{(}\PY{l+s+sa}{f}\PY{l+s+s2}{\PYZdq{}}\PY{l+s+s2}{The minimum is }\PY{l+s+si}{\PYZob{}}\PY{n}{min\PYZus{}value}\PY{l+s+si}{\PYZcb{}}\PY{l+s+s2}{. The maximum is }\PY{l+s+si}{\PYZob{}}\PY{n}{max\PYZus{}value}\PY{l+s+si}{\PYZcb{}}\PY{l+s+s2}{.}\PY{l+s+s2}{\PYZdq{}}\PY{p}{)}
\end{Verbatim}
\end{tcolorbox}

    \hypertarget{task-4}{%
\subsection{Task 4}\label{task-4}}

(2 PTS)

Write a lambda function that applies the following function to \(x\):

\begin{quote}
\(log_{10}(1 + x)\)
\end{quote}

Return the rounded value to \(2\) decimals.

Assign the function to the variable \texttt{mylog10}.

Then call the lambda function on \texttt{area} and print the last 10
values.

Hints: * Use the \texttt{log10} function from Python's
\href{https://docs.python.org/3/library/math.html}{\texttt{math}
module}. You'll need to import it. * Use a list comprehension to make
the lambda function a one-liner. * To get the last members of a list,
used negative offset slicing. See
\href{https://docs.python.org/3/tutorial/introduction.html\#lists}{the
Python documentation on lists} for a refresher on slicing.

    \begin{tcolorbox}[breakable, size=fbox, boxrule=1pt, pad at break*=1mm,colback=cellbackground, colframe=cellborder]
\prompt{In}{incolor}{38}{\boxspacing}
\begin{Verbatim}[commandchars=\\\{\}]
\PY{k+kn}{import} \PY{n+nn}{math}
\end{Verbatim}
\end{tcolorbox}

    \begin{tcolorbox}[breakable, size=fbox, boxrule=1pt, pad at break*=1mm,colback=cellbackground, colframe=cellborder]
\prompt{In}{incolor}{43}{\boxspacing}
\begin{Verbatim}[commandchars=\\\{\}]
\PY{n}{mylog10} \PY{o}{=} \PY{k}{lambda} \PY{n}{x}\PY{p}{:} \PY{n+nb}{round}\PY{p}{(}\PY{n}{math}\PY{o}{.}\PY{n}{log10}\PY{p}{(}\PY{l+m+mi}{1}\PY{o}{+}\PY{n}{x}\PY{p}{)}\PY{p}{,} \PY{l+m+mi}{2}\PY{p}{)}
\end{Verbatim}
\end{tcolorbox}

    \begin{tcolorbox}[breakable, size=fbox, boxrule=1pt, pad at break*=1mm,colback=cellbackground, colframe=cellborder]
\prompt{In}{incolor}{51}{\boxspacing}
\begin{Verbatim}[commandchars=\\\{\}]

\end{Verbatim}
\end{tcolorbox}

            \begin{tcolorbox}[breakable, size=fbox, boxrule=.5pt, pad at break*=1mm, opacityfill=0]
\prompt{Out}{outcolor}{51}{\boxspacing}
\begin{Verbatim}[commandchars=\\\{\}]
<function \_\_main\_\_.<lambda>(x)>
\end{Verbatim}
\end{tcolorbox}
        
    \begin{tcolorbox}[breakable, size=fbox, boxrule=1pt, pad at break*=1mm,colback=cellbackground, colframe=cellborder]
\prompt{In}{incolor}{53}{\boxspacing}
\begin{Verbatim}[commandchars=\\\{\}]
\PY{p}{[}\PY{n+nb}{round}\PY{p}{(}\PY{n}{math}\PY{o}{.}\PY{n}{log10}\PY{p}{(}\PY{l+m+mi}{1}\PY{o}{+}\PY{n}{x}\PY{p}{)}\PY{p}{,} \PY{l+m+mi}{2}\PY{p}{)} \PY{k}{for} \PY{n}{x} \PY{o+ow}{in} \PY{n}{area}\PY{p}{[}\PY{o}{\PYZhy{}}\PY{l+m+mi}{10}\PY{p}{:}\PY{p}{]}\PY{p}{]}
\end{Verbatim}
\end{tcolorbox}

            \begin{tcolorbox}[breakable, size=fbox, boxrule=.5pt, pad at break*=1mm, opacityfill=0]
\prompt{Out}{outcolor}{53}{\boxspacing}
\begin{Verbatim}[commandchars=\\\{\}]
[0.0, 0.0, 0.5, 0.16, 0.0, 0.87, 1.74, 1.08, 0.0, 0.0]
\end{Verbatim}
\end{tcolorbox}
        
    \hypertarget{working-with-month}{%
\section{Working with MONTH}\label{working-with-month}}

The month column contains months of the year in abbreviated form ---
\texttt{jan} to \texttt{dec}.

    \begin{tcolorbox}[breakable, size=fbox, boxrule=1pt, pad at break*=1mm,colback=cellbackground, colframe=cellborder]
\prompt{In}{incolor}{57}{\boxspacing}
\begin{Verbatim}[commandchars=\\\{\}]
\PY{n}{month} \PY{o}{=} \PY{n}{firedata}\PY{p}{[}\PY{l+s+s1}{\PYZsq{}}\PY{l+s+s1}{month}\PY{l+s+s1}{\PYZsq{}}\PY{p}{]}
\end{Verbatim}
\end{tcolorbox}

    \begin{tcolorbox}[breakable, size=fbox, boxrule=1pt, pad at break*=1mm,colback=cellbackground, colframe=cellborder]
\prompt{In}{incolor}{19}{\boxspacing}
\begin{Verbatim}[commandchars=\\\{\}]
\PY{n}{month}\PY{p}{[}\PY{p}{:}\PY{l+m+mi}{10}\PY{p}{]}
\end{Verbatim}
\end{tcolorbox}

            \begin{tcolorbox}[breakable, size=fbox, boxrule=.5pt, pad at break*=1mm, opacityfill=0]
\prompt{Out}{outcolor}{19}{\boxspacing}
\begin{Verbatim}[commandchars=\\\{\}]
['mar', 'oct', 'oct', 'mar', 'mar', 'aug', 'aug', 'aug', 'sep', 'sep']
\end{Verbatim}
\end{tcolorbox}
        
    \hypertarget{task-5}{%
\subsection{Task 5}\label{task-5}}

(1 PT)

Create a function called \texttt{get\_uniques()} that extracts the
unique values from a list. * Do not use \texttt{set()} but instead use a
\textbf{dictionary comprehension} to capture the unique names. * Hint:
They keys in a dictionary are unique. * Hint: You do not need to count
how many times a name appears in the source list.

Then function should optionally return the list as sorted in ascending
order.

Then apply it to the \texttt{month} column of our data with sorting
turned on.

Then print the unique months.

    \begin{tcolorbox}[breakable, size=fbox, boxrule=1pt, pad at break*=1mm,colback=cellbackground, colframe=cellborder]
\prompt{In}{incolor}{92}{\boxspacing}
\begin{Verbatim}[commandchars=\\\{\}]
\PY{k}{def} \PY{n+nf}{get\PYZus{}uniques}\PY{p}{(}\PY{n}{ulist}\PY{p}{)}\PY{p}{:}
    \PY{n}{uniquelist} \PY{o}{=} \PY{p}{[}\PY{p}{]}
    \PY{p}{[}\PY{n}{uniquelist}\PY{o}{.}\PY{n}{append}\PY{p}{(}\PY{n}{x}\PY{p}{)} \PY{k}{for} \PY{n}{x} \PY{o+ow}{in} \PY{n}{ulist} \PY{k}{if} \PY{o+ow}{not} \PY{n}{x} \PY{o+ow}{in} \PY{n}{uniquelist}\PY{p}{]}
    \PY{k}{return}\PY{p}{(}\PY{n}{uniquelist}\PY{p}{)}
\end{Verbatim}
\end{tcolorbox}

    \begin{tcolorbox}[breakable, size=fbox, boxrule=1pt, pad at break*=1mm,colback=cellbackground, colframe=cellborder]
\prompt{In}{incolor}{93}{\boxspacing}
\begin{Verbatim}[commandchars=\\\{\}]
\PY{n}{get\PYZus{}uniques}\PY{p}{(}\PY{n}{month}\PY{p}{)}
\end{Verbatim}
\end{tcolorbox}

            \begin{tcolorbox}[breakable, size=fbox, boxrule=.5pt, pad at break*=1mm, opacityfill=0]
\prompt{Out}{outcolor}{93}{\boxspacing}
\begin{Verbatim}[commandchars=\\\{\}]
['mar',
 'oct',
 'aug',
 'sep',
 'apr',
 'jun',
 'jul',
 'feb',
 'jan',
 'dec',
 'may',
 'nov']
\end{Verbatim}
\end{tcolorbox}
        
    \hypertarget{task-6}{%
\subsection{Task 6}\label{task-6}}

(1 PT)

Write a lambda function called \texttt{get\_month\_for\_letter} that
uses a list comprehension to select all months starting with a given
letter from the list of unique month names you just crreated.

The function should assume that the list of unique month names exists in
the global context.

The returned list should contain uppercase strings.

Run and print the result with \texttt{a} as the paramter.

    \begin{tcolorbox}[breakable, size=fbox, boxrule=1pt, pad at break*=1mm,colback=cellbackground, colframe=cellborder]
\prompt{In}{incolor}{106}{\boxspacing}
\begin{Verbatim}[commandchars=\\\{\}]
\PY{k}{def} \PY{n+nf}{get\PYZus{}month\PYZus{}for\PYZus{}letter}\PY{p}{(}\PY{n}{months}\PY{p}{,} \PY{n}{letter}\PY{p}{)}\PY{p}{:}
    \PY{k}{return} \PY{p}{[}\PY{n}{x}\PY{o}{.}\PY{n}{upper}\PY{p}{(}\PY{p}{)} \PY{k}{for} \PY{n}{x} \PY{o+ow}{in} \PY{n}{months} \PY{k}{if} \PY{n}{x}\PY{p}{[}\PY{p}{:}\PY{l+m+mi}{1}\PY{p}{]} \PY{o}{==} \PY{n}{letter}\PY{p}{]}
    
\end{Verbatim}
\end{tcolorbox}

    \hypertarget{working-with-dmc}{%
\section{Working with DMC}\label{working-with-dmc}}

DMC - DMC index from the FWI system: 1.1 to 291.3

    \begin{tcolorbox}[breakable, size=fbox, boxrule=1pt, pad at break*=1mm,colback=cellbackground, colframe=cellborder]
\prompt{In}{incolor}{109}{\boxspacing}
\begin{Verbatim}[commandchars=\\\{\}]
\PY{n}{dmc} \PY{o}{=} \PY{n}{firedata}\PY{p}{[}\PY{l+s+s1}{\PYZsq{}}\PY{l+s+s1}{DMC}\PY{l+s+s1}{\PYZsq{}}\PY{p}{]}
\end{Verbatim}
\end{tcolorbox}

    \begin{tcolorbox}[breakable, size=fbox, boxrule=1pt, pad at break*=1mm,colback=cellbackground, colframe=cellborder]
\prompt{In}{incolor}{110}{\boxspacing}
\begin{Verbatim}[commandchars=\\\{\}]
\PY{n}{dmc}\PY{p}{[}\PY{p}{:}\PY{l+m+mi}{10}\PY{p}{]}
\end{Verbatim}
\end{tcolorbox}

            \begin{tcolorbox}[breakable, size=fbox, boxrule=.5pt, pad at break*=1mm, opacityfill=0]
\prompt{Out}{outcolor}{110}{\boxspacing}
\begin{Verbatim}[commandchars=\\\{\}]
[26.2, 35.4, 43.7, 33.3, 51.3, 85.3, 88.9, 145.4, 129.5, 88.0]
\end{Verbatim}
\end{tcolorbox}
        
    \hypertarget{task-7}{%
\subsection{Task 7}\label{task-7}}

(2 PTS)

Write a function called \texttt{bandpass\_filter()} with these
requirements:

\begin{itemize}
\tightlist
\item
  Takes three inputs:

  \begin{itemize}
  \tightlist
  \item
    A list of numbers \texttt{num\_list}.
  \item
    An integer serving as a lower bound \texttt{lower\_bound}.
  \item
    An integer serving as an upper bound \texttt{upper\_bound}.
  \end{itemize}
\item
  Returns a new array containing only the values from the original array
  which are greater than \texttt{lower\_bound} and less than
  \texttt{upper\_bound}.
\end{itemize}

    \begin{tcolorbox}[breakable, size=fbox, boxrule=1pt, pad at break*=1mm,colback=cellbackground, colframe=cellborder]
\prompt{In}{incolor}{111}{\boxspacing}
\begin{Verbatim}[commandchars=\\\{\}]
\PY{k}{def} \PY{n+nf}{bandpass\PYZus{}filter}\PY{p}{(}\PY{n}{num\PYZus{}list}\PY{p}{,} \PY{n}{lower\PYZus{}bound}\PY{p}{,} \PY{n}{upper\PYZus{}bound}\PY{p}{)}\PY{p}{:}
    \PY{k}{return} \PY{p}{[}\PY{n}{x} \PY{k}{for} \PY{n}{x} \PY{o+ow}{in} \PY{n}{num\PYZus{}list} \PY{k}{if} \PY{n}{x} \PY{o}{\PYZgt{}} \PY{n}{lower\PYZus{}bound} \PY{o+ow}{and} \PY{n}{x} \PY{o}{\PYZlt{}} \PY{n}{upper\PYZus{}bound}\PY{p}{]}
\end{Verbatim}
\end{tcolorbox}

    \hypertarget{task-8}{%
\subsection{Task 8}\label{task-8}}

(1 PT)

Call \texttt{bandpass\_filter()} passing \texttt{dmc} as the list, with
\texttt{lower\_bound=25} and \texttt{upper\_bound=35}.

Then print the result.

    \begin{tcolorbox}[breakable, size=fbox, boxrule=1pt, pad at break*=1mm,colback=cellbackground, colframe=cellborder]
\prompt{In}{incolor}{115}{\boxspacing}
\begin{Verbatim}[commandchars=\\\{\}]
\PY{n}{bandpass\PYZus{}filter}\PY{p}{(}\PY{n+nb}{list}\PY{p}{(}\PY{n}{dmc}\PY{p}{)}\PY{p}{,}\PY{l+m+mi}{25}\PY{p}{,}\PY{l+m+mi}{35}\PY{p}{)}
\end{Verbatim}
\end{tcolorbox}

            \begin{tcolorbox}[breakable, size=fbox, boxrule=.5pt, pad at break*=1mm, opacityfill=0]
\prompt{Out}{outcolor}{115}{\boxspacing}
\begin{Verbatim}[commandchars=\\\{\}]
[26.2,
 33.3,
 32.8,
 27.9,
 27.4,
 25.7,
 33.3,
 33.3,
 30.7,
 33.3,
 25.7,
 25.7,
 25.7,
 32.8,
 27.2,
 27.8,
 26.4,
 25.4,
 25.4,
 25.4,
 25.4,
 26.7,
 25.4,
 27.5,
 28.0,
 25.4]
\end{Verbatim}
\end{tcolorbox}
        
    \hypertarget{working-with-ffmc}{%
\section{Working with FFMC}\label{working-with-ffmc}}

FFMC - FFMC index from the FWI system: 18.7 to 96.20

    \begin{tcolorbox}[breakable, size=fbox, boxrule=1pt, pad at break*=1mm,colback=cellbackground, colframe=cellborder]
\prompt{In}{incolor}{116}{\boxspacing}
\begin{Verbatim}[commandchars=\\\{\}]
\PY{n}{ffmc} \PY{o}{=} \PY{n}{firedata}\PY{p}{[}\PY{l+s+s1}{\PYZsq{}}\PY{l+s+s1}{FFMC}\PY{l+s+s1}{\PYZsq{}}\PY{p}{]}
\end{Verbatim}
\end{tcolorbox}

    \begin{tcolorbox}[breakable, size=fbox, boxrule=1pt, pad at break*=1mm,colback=cellbackground, colframe=cellborder]
\prompt{In}{incolor}{117}{\boxspacing}
\begin{Verbatim}[commandchars=\\\{\}]
\PY{n}{ffmc}\PY{p}{[}\PY{p}{:}\PY{l+m+mi}{10}\PY{p}{]}
\end{Verbatim}
\end{tcolorbox}

            \begin{tcolorbox}[breakable, size=fbox, boxrule=.5pt, pad at break*=1mm, opacityfill=0]
\prompt{Out}{outcolor}{117}{\boxspacing}
\begin{Verbatim}[commandchars=\\\{\}]
[86.2, 90.6, 90.6, 91.7, 89.3, 92.3, 92.3, 91.5, 91.0, 92.5]
\end{Verbatim}
\end{tcolorbox}
        
    \hypertarget{task-9}{%
\subsection{Task 9}\label{task-9}}

(2 PTS)

Write a lambda function \texttt{get\_mean} that computes the mean
\(\mu\) of a list of numbers. * The mean is jus the sum of a list of
numeric values divided by the length of that list.

Write another lambda function \texttt{get\_ssd} that computes the
squared deviation of a number. * The function takes two arguments, a
number from a given list and the mean of the numbers in that list. * The
function is meant to be used in a for-loop that iterates through a list.
* The squared deviation of a list element \(x_i\) is \((x_i - \mu)^2\).

Then write \texttt{get\_sum\_sq\_err()} with these requirements: * Takes
a numeric list as input. * Computes the mean \(\mu\) of the list using
\texttt{get\_mean}. * Computes the sum of squared deviations for the
list using a list comprehension that applies \texttt{get\_ssd}. *
Returns the sum of squared deviations.

    \begin{tcolorbox}[breakable, size=fbox, boxrule=1pt, pad at break*=1mm,colback=cellbackground, colframe=cellborder]
\prompt{In}{incolor}{134}{\boxspacing}
\begin{Verbatim}[commandchars=\\\{\}]
\PY{n}{get\PYZus{}mean} \PY{o}{=} \PY{k}{lambda} \PY{n}{x}\PY{p}{:} \PY{n+nb}{sum}\PY{p}{(}\PY{n}{x}\PY{p}{)}\PY{o}{/}\PY{n+nb}{len}\PY{p}{(}\PY{n}{x}\PY{p}{)}
\PY{n}{get\PYZus{}ssd} \PY{o}{=} \PY{k}{lambda} \PY{n}{x}\PY{p}{,} \PY{n}{y}\PY{p}{:} \PY{p}{(}\PY{n}{x} \PY{o}{\PYZhy{}} \PY{n}{y}\PY{p}{)}\PY{o}{*}\PY{o}{*}\PY{l+m+mi}{2}
\PY{k}{def} \PY{n+nf}{get\PYZus{}sum\PYZus{}sq\PYZus{}err}\PY{p}{(}\PY{n}{numlist}\PY{p}{)}\PY{p}{:}
    \PY{n}{mean} \PY{o}{=} \PY{n}{get\PYZus{}mean}\PY{p}{(}\PY{n}{numlist}\PY{p}{)}
    \PY{k}{return} \PY{p}{[}\PY{n}{get\PYZus{}ssd}\PY{p}{(}\PY{n}{x}\PY{p}{,} \PY{n}{mean}\PY{p}{)} \PY{k}{for} \PY{n}{x} \PY{o+ow}{in} \PY{n}{numlist}\PY{p}{]}
\end{Verbatim}
\end{tcolorbox}

    \hypertarget{task-10}{%
\subsection{Task 10}\label{task-10}}

(1 PT)

Call \texttt{sum\_sq\_err()} passing \texttt{ffmc} as the list and print
the result.

    \begin{tcolorbox}[breakable, size=fbox, boxrule=1pt, pad at break*=1mm,colback=cellbackground, colframe=cellborder]
\prompt{In}{incolor}{136}{\boxspacing}
\begin{Verbatim}[commandchars=\\\{\}]
\PY{n}{get\PYZus{}sum\PYZus{}sq\PYZus{}err}\PY{p}{(}\PY{n}{ffmc}\PY{p}{)}
\end{Verbatim}
\end{tcolorbox}

            \begin{tcolorbox}[breakable, size=fbox, boxrule=.5pt, pad at break*=1mm, opacityfill=0]
\prompt{Out}{outcolor}{136}{\boxspacing}
\begin{Verbatim}[commandchars=\\\{\}]
[19.7551878678114,
 0.0019963784517678914,
 0.0019963784517678914,
 1.113698506111859,
 1.8081665912171285,
 2.7400814848355264,
 2.7400814848355264,
 0.7315708465372919,
 0.12625169760088936,
 3.442209144410097,
 3.442209144410097,
 4.645400633771926,
 736.8336985060987,
 0.06518786781361177,
 5.086464463559245,
 7.050719782708326,
 1.113698506111859,
 33.001358080576715,
 2.0871027614298323,
 18.876251697598732,
 0.12625169760088936,
 1.3347623358991267,
 13.361358080581125,
 0.19774105930264288,
 8.152847442282901,
 0.5705070167500199,
 3.0811453146228365,
 0.06518786781361177,
 7.591783612495653,
 8.152847442282901,
 13.361358080581125,
 4.180719782706181,
 4.180719782706181,
 1.113698506111859,
 1.3347623358991267,
 0.11880488908992785,
 0.0019963784517678914,
 0.4156133997280844,
 0.0019963784517678914,
 6.475400633769784,
 124.20391127206364,
 0.19774105930264288,
 17.266677229517526,
 3.442209144410097,
 0.29667722951537107,
 13.361358080581125,
 0.06518786781361177,
 12.640294250793886,
 11.865826165684211,
 9.270081484833387,
 5.086464463559245,
 0.19774105930264288,
 2.117953825260958,
 2.117953825260958,
 1.113698506111859,
 5.086464463559245,
 0.11880488908992785,
 3.823272974197355,
 44.15178361249126,
 16.359443186960593,
 1.8081665912171285,
 1.8081665912171285,
 5.547528293346499,
 0.19774105930264288,
 0.2073155273881647,
 1.113698506111859,
 3.0811453146228365,
 3.0811453146228365,
 3.0811453146228365,
 1.113698506111859,
 0.3083793571754535,
 13.361358080581125,
 1.113698506111859,
 3.402847442280729,
 7.050719782708326,
 41.53391127206578,
 16.359443186960593,
 9.270081484833387,
 0.29667722951537107,
 0.12625169760088936,
 0.5705070167500199,
 0.19774105930264288,
 17.266677229517526,
 2.117953825260958,
 1.113698506111859,
 5.086464463559245,
 5.086464463559245,
 5.086464463559245,
 8.152847442282901,
 1.113698506111859,
 0.19774105930264288,
 1.113698506111859,
 2.7400814848355264,
 0.5705070167500199,
 0.2073155273881647,
 0.8924219103662375,
 45.4907197827039,
 468.49220914439917,
 0.5705070167500199,
 0.5705070167500199,
 0.5705070167500199,
 3.402847442280729,
 17.266677229517526,
 3.442209144410097,
 73.01157084653302,
 22.51199637844953,
 0.5705070167500199,
 0.19774105930264288,
 3.442209144410097,
 4.180719782706181,
 22.51199637844953,
 1.113698506111859,
 0.8924219103662375,
 1.3347623358991267,
 6.475400633769784,
 6.475400633769784,
 1.113698506111859,
 1.113698506111859,
 0.29667722951537107,
 5.547528293346499,
 0.7315708465372919,
 0.7315708465372919,
 3.0811453146228365,
 38.99603893164031,
 13.361358080581125,
 3.823272974197355,
 9.270081484833387,
 8.152847442282901,
 0.5705070167500199,
 3.823272974197355,
 503.7636985061008,
 11.865826165684211,
 1.8081665912171285,
 9.33497510185748,
 6.475400633769784,
 8.152847442282901,
 3.0811453146228365,
 0.06518786781361177,
 23.470932548662333,
 0.12625169760088936,
 0.06518786781361177,
 23.57412403802851,
 0.29667722951537107,
 0.4156133997280844,
 23.57412403802851,
 20.750932548666697,
 0.29667722951537107,
 38.99603893164031,
 17.266677229517526,
 9.33497510185748,
 3.442209144410097,
 0.29667722951537107,
 0.29667722951537107,
 13.361358080581125,
 7.591783612495653,
 17.266677229517526,
 7.591783612495653,
 2.117953825260958,
 2.117953825260958,
 3.0811453146228365,
 0.29667722951537107,
 20.750932548666697,
 0.0019963784517678914,
 3.442209144410097,
 0.8924219103662375,
 33.001358080576715,
 2.117953825260958,
 28.679443186964914,
 0.3083793571754535,
 20.750932548666697,
 5.086464463559245,
 25.4488048890878,
 0.5705070167500199,
 0.06518786781361177,
 0.19774105930264288,
 0.19774105930264288,
 83.62518786780925,
 0.19774105930264288,
 0.29667722951537107,
 3.402847442280729,
 3.0811453146228365,
 33.001358080576715,
 14.78157084653513,
 1.113698506111859,
 10.59710276143206,
 0.12625169760088936,
 1.8081665912171285,
 0.06518786781361177,
 0.024124038026327493,
 0.0030602082390481945,
 0.3083793571754535,
 20.750932548666697,
 17.266677229517526,
 5.086464463559245,
 17.266677229517526,
 10.59710276143206,
 83.62518786780925,
 5.086464463559245,
 0.12625169760088936,
 736.8336985060987,
 0.12625169760088936,
 0.29667722951537107,
 45.4907197827039,
 0.5705070167500199,
 0.0019963784517678914,
 5.086464463559245,
 8.152847442282901,
 0.12625169760088936,
 3.0811453146228365,
 9.33497510185748,
 0.0019963784517678914,
 8.152847442282901,
 13.361358080581125,
 9.270081484833387,
 1.113698506111859,
 1.113698506111859,
 3.0811453146228365,
 4.180719782706181,
 5.086464463559245,
 0.29667722951537107,
 0.19774105930264288,
 7.050719782708326,
 9.270081484833387,
 5.497528293344329,
 0.29667722951537107,
 8.152847442282901,
 1.113698506111859,
 2.4190176550482643,
 8.152847442282901,
 2.4190176550482643,
 5.086464463559245,
 8.152847442282901,
 0.12625169760088936,
 38.99603893164031,
 3.442209144410097,
 0.5705070167500199,
 3.442209144410097,
 0.12625169760088936,
 3.442209144410097,
 76.46944318695826,
 6.994336803982474,
 58.44114531461845,
 12.640294250793886,
 1.3347623358991267,
 1.3347623358991267,
 1.3347623358991267,
 1.3347623358991267,
 2.4190176550482643,
 6.028592123133752,
 6.028592123133752,
 6.028592123133752,
 6.028592123133752,
 6.028592123133752,
 1.575826165686428,
 0.9126346763245615,
 9.889017655046072,
 12.640294250793886,
 12.640294250793886,
 1.3347623358991267,
 1.3347623358991267,
 8.733911272070149,
 0.9126346763245615,
 0.9126346763245615,
 0.2073155273881647,
 13.361358080581125,
 9.33497510185748,
 13.361358080581125,
 2.117953825260958,
 2.117953825260958,
 2.117953825260958,
 2.117953825260958,
 2.117953825260958,
 2.117953825260958,
 38.99603893164031,
 44.15178361249126,
 36.53816659121501,
 27.506677229513123,
 27.506677229513123,
 27.506677229513123,
 27.506677229513123,
 35.339230421002185,
 27.506677229513123,
 33.001358080576715,
 14.022634676322346,
 29.64454956993859,
 10.59710276143206,
 0.3083793571754535,
 0.9126346763245615,
 0.9126346763245615,
 0.9126346763245615,
 0.9126346763245615,
 0.9126346763245615,
 0.9126346763245615,
 6.028592123133752,
 2.7400814848355264,
 6.028592123133752,
 0.05986871887720361,
 0.05986871887720361,
 0.3083793571754535,
 1387.1662516975834,
 0.05986871887720361,
 0.05986871887720361,
 0.2073155273881647,
 0.2073155273881647,
 30.743485740151403,
 1.0913580805789742,
 1.0913580805789742,
 1.0913580805789742,
 3.0811453146228365,
 3.0811453146228365,
 3.0811453146228365,
 3.0811453146228365,
 1619.6343368039652,
 3.823272974197355,
 3.823272974197355,
 0.3083793571754535,
 3.0811453146228365,
 3.0811453146228365,
 4.645400633771926,
 4.645400633771926,
 4.645400633771926,
 4.645400633771926,
 4.645400633771926,
 0.0030602082390481945,
 6.475400633769784,
 2.4190176550482643,
 2.4190176550482643,
 2.4190176550482643,
 2.4190176550482643,
 2.4190176550482643,
 2.4190176550482643,
 0.3083793571754535,
 0.3083793571754535,
 2.117953825260958,
 0.9126346763245615,
 0.9126346763245615,
 0.9126346763245615,
 0.9126346763245615,
 0.9126346763245615,
 0.9126346763245615,
 0.9126346763245615,
 1.575826165686428,
 0.7315708465372919,
 0.7315708465372919,
 0.7315708465372919,
 0.7315708465372919,
 0.7315708465372919,
 2.117953825260958,
 2.117953825260958,
 2.117953825260958,
 2.117953825260958,
 2.117953825260958,
 2.117953825260958,
 2.117953825260958,
 2.117953825260958,
 2.117953825260958,
 2.117953825260958,
 3.442209144410097,
 3.442209144410097,
 3.442209144410097,
 3.442209144410097,
 3.442209144410097,
 5.976464463557021,
 1.575826165686428,
 1.575826165686428,
 1.575826165686428,
 0.2073155273881647,
 0.2073155273881647,
 0.3083793571754535,
 0.12625169760088936,
 0.12625169760088936,
 1.575826165686428,
 1.8368899954736944,
 17.266677229517526,
 0.11880488908992785,
 0.3083793571754535,
 2.117953825260958,
 9.33497510185748,
 0.06518786781361177,
 5176.037102761396,
 9.33497510185748,
 0.0030602082390481945,
 20.750932548666697,
 0.9126346763245615,
 0.9126346763245615,
 0.9126346763245615,
 0.020932548664486888,
 0.06518786781361177,
 17.266677229517526,
 17.266677229517526,
 35.339230421002185,
 0.2073155273881647,
 0.12625169760088936,
 7.591783612495653,
 42.83284744227861,
 31.86242191036406,
 0.020932548664486888,
 0.9126346763245615,
 9.33497510185748,
 7.050719782708326,
 7.050719782708326,
 0.2073155273881647,
 17.266677229517526,
 0.0030602082390481945,
 7.533272974195163,
 15.644549569942942,
 12.56476233589699,
 35.339230421002185,
 0.11880488908992785,
 2.7400814848355264,
 42.83284744227861,
 36.53816659121501,
 2.7400814848355264,
 9.33497510185748,
 8.733911272070149,
 17.266677229517526,
 4.224336803984669,
 7.591783612495653,
 1.8368899954736944,
 0.9126346763245615,
 1.113698506111859,
 20.750932548666697,
 3.0439112720679793,
 0.3083793571754535,
 9.33497510185748,
 0.9126346763245615,
 0.9126346763245615,
 2.117953825260958,
 17.266677229517526,
 0.9126346763245615,
 0.8924219103662375,
 2.117953825260958,
 17.266677229517526,
 8.733911272070149,
 0.0019963784517678914,
 0.024124038026327493,
 2.117953825260958,
 1.5492304210043872,
 9.33497510185748,
 0.2073155273881647,
 0.11880488908992785,
 2.117953825260958,
 7.533272974195163,
 0.0030602082390481945,
 0.11880488908992785,
 11.258166591219304,
 1.8368899954736944,
 7.591783612495653,
 0.8924219103662375,
 0.9126346763245615,
 20.750932548666697,
 0.020932548664486888,
 0.7315708465372919,
 1.5492304210043872,
 0.9126346763245615,
 15.644549569942942,
 0.9126346763245615,
 1.113698506111859,
 9.33497510185748,
 0.9126346763245615,
 9.33497510185748,
 9.33497510185748,
 0.12625169760088936,
 241.63710276142348,
 241.63710276142348,
 124.20391127206364,
 11.865826165684211,
 0.19774105930264288,
 0.42944318696272715,
 0.12625169760088936,
 0.12625169760088936,
 1.0913580805789742,
 5.976464463557021,
 0.020932548664486888,
 5.547528293346499,
 9.33497510185748,
 8.152847442282901,
 9.33497510185748,
 9.33497510185748,
 2.0871027614298323,
 6.5296559529210745,
 6.5296559529210745,
 18.107741059304878,
 18.107741059304878,
 18.107741059304878,
 18.96880488909211,
 19.84986871887934,
 19.84986871887934,
 19.84986871887934,
 19.84986871887934,
 19.84986871887934,
 26.577315527390326,
 27.618379357177695,
 27.618379357177695,
 28.679443186964914,
 30.861570846539507,
 30.861570846539507,
 29.760507016752133,
 29.760507016752133,
 29.760507016752133,
 29.760507016752133,
 29.760507016752133,
 29.760507016752133,
 14.863485740155706,
 14.863485740155706,
 0.12625169760088936,
 0.12625169760088936,
 0.12625169760088936,
 0.12625169760088936,
 0.12625169760088936,
 0.12625169760088936,
 81.80625169759662,
 81.80625169759662,
 81.80625169759662,
 81.80625169759662,
 14.102421910368468,
 124.20391127206364]
\end{Verbatim}
\end{tcolorbox}
        
    \begin{tcolorbox}[breakable, size=fbox, boxrule=1pt, pad at break*=1mm,colback=cellbackground, colframe=cellborder]
\prompt{In}{incolor}{ }{\boxspacing}
\begin{Verbatim}[commandchars=\\\{\}]

\end{Verbatim}
\end{tcolorbox}


    % Add a bibliography block to the postdoc
    
    
    
\end{document}
